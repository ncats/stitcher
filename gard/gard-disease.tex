\documentclass[anchorcolor=blue,linkcolor=blue]{beamer}
\usepackage{booktabs}
\usepackage{amsmath}
\usepackage{xcolor}
%\usepackage{tabularx}
\usepackage{rotating}

\hypersetup{colorlinks,linkcolor=,urlcolor=blue}

%%\usepackage{blkarray}
\mode<presentation>
{
%  \usetheme{Malmoe}
\usetheme{default}
%\usecolortheme{seahorse}
  % or ...

% \setbeamercovered{transparent}
  % or whatever (possibly just delete it)
% \setbeamertemplate{footline}[default]
% \setbeamertemplate{navigation symbols}{\insertslidenavigationsymbol\insertframenavigationsymbol\insertdocnavigationsymbol}
}
\usepackage[english]{babel}

\title{How many (rare) diseases are there?}
\subtitle{A feeble attempt at systematic disease\\
  harmonization across ontologies} 
\author{Dac-Trung Nguyen\and Qian Zhu\\[1em] \emph{NCATS Informatics}}
%\date{{\Large$\pi\!$}, 2019}
\date{Jan 6, 2020}

\begin{document}

\begin{frame}
  \titlepage
\end{frame}

\begin{frame}
  \begin{block}{How dare you?}
    \begin{itemize}
    \item No agreed upon definition for what a ``disease'' is
    \item \emph{disease} $\rightleftharpoons$ \emph{syndrome}
      $\rightleftharpoons$ \emph{condition} $\rightleftharpoons$
      \emph{indication} $\cdots$
      \begin{itemize}
        \item Andersen Syndrome (C1563715) vs. Andersen's Disease (C0017923); both are cross referenced by Orphanet 367.
      \end{itemize}
    \item Numerous ontologies: MeSH, OMIM, Disease Ontology, Orphanet,
      GARD, UMLS, HPO, MONDO, $\ldots$
    \item Entity resolution/normalization and ontology matching are
      still unsolved problems
    \end{itemize}
  \end{block}
  \begin{block}{Baby steps...}
    \begin{enumerate}
      \item Build a comprehensive knowledge graph based on available
        ontologies: \href{https://www.ebi.ac.uk/ols/ontologies}{Ontology
          Lookup Service}
      \item Build on our previous effort for
        \href{https://stitcher.ncats.io/app/stitches/latest}{drug harmonization} using
        in-house
        \href{https://github.com/ncats/stitcher}{stitcher} codebase
      \item Focus on the GARD subset
    \end{enumerate}
  \end{block}
\end{frame}

\begin{frame}
  \frametitle{Disease knowledge graph}
  \begin{itemize}
  \item Available as a Neo4j database at
    
    \centerline{\href{https://disease.ncats.io/browser}{\texttt{https://disease.ncats.io/browser}}}
    \begin{itemize}
    \item No username/password required
    \item Use \texttt{disease.ncats.io:80} as the \textbf{Host}
    \end{itemize}
  \item 1,806,027 entities spanning diseases, genes, proteins, drugs,
    phenotypes, tissues, FDA orphan designations.
  \item 473,382,893 relationships that span ontological (e.g.,
    \texttt{subclassOf}), phenotypic (e.g.,
    \texttt{manifestation\_of}), disease association (e.g.,
    \texttt{is\_associated\_disease\_of}), etc.
  \item Accessible programmatically via numerous Neo4j drivers
    \begin{itemize}
      \item Python 2.7 example \href{https://github.com/ncats/gitlab-snippets/blob/master/neo4j-disease.py}{https://github.com/ncats/gitlab-snippets/blob/master/neo4j-disease.py}
    \end{itemize}
  \end{itemize}
\end{frame}

\begin{frame}
  \frametitle{Disease knowledge graph data sources}
  \begin{center}
    \tiny
    \begin{tabular}{rl}\toprule
      Data Source & Entities\\ \midrule
      \texttt{GARD}&6,763\\
      \texttt{BRENDATISSUE}&5,902\\
      \texttt{GHR}&1,287\\
      \texttt{DOID}&12,694\\
      \texttt{HPO}&17,387\\
      \texttt{MEDLINEPLUS}&2,238\\
      \texttt{MESH}&275,329\\
      \texttt{MONDO}&110,363\\
      \texttt{OMIM}&102,715\\
      \texttt{UBERON}&15,909\\
      \texttt{ORDO}&13,871\\
      \texttt{GO}&49,290\\
      \texttt{OGG}&69,688\\
      \texttt{PR}&315,937\\
      \texttt{OGMS}&91\\
      \texttt{PATO}&2,730\\
      \texttt{CHEBI}&130,403\\
      \texttt{FDAOrphanGARD\_20190216.txt}&6,074\\
      \texttt{RANCHO-DISEASE-DRUG\_2018-12-18\_13-30}&19,817\\
      \texttt{HPO\_ANNOTATION\_100918}&164,448\\
      \texttt{MEDGEN} &216,700\\
      \texttt{ICD10} &94,645 \\
      \texttt{NCI Thesaurus} & 144,695\\
      \texttt{EFO} & 26,300\\ \bottomrule
    \end{tabular}
  \end{center}
\end{frame}

\begin{frame}
  \frametitle{A closer look at GARD}
    
  \begin{block}{GARD diseases by content feature}
    \begin{center}
      \begin{tabular}{cc}\toprule
        Content Feature & Count\\ \midrule
        Prevalence & 95 \\
        Diagnosis & 577\\
        Inheritance & 702\\
        Treatment & 1037\\
        Cause & 873\\
        Symptoms & 835\\
        Prognosis & 568\\ \bottomrule
      \end{tabular}
    \end{center}
    Total 6,763 diseases, of which 6,504 are considered rare. Out of
    6,504 rare diseases, there are 480 that do not map to anything (e.g.,
    \emph{Hillig syndrome}). It turns out that about 58 of these do map
    to UMLS exactly; e.g., \emph{Webster Deming syndrome} (GARD 428)
    maps to \emph{Craniofrontonasal dysplasia with Poland anomaly
      syndrome} (C4303859).
  \end{block}
\end{frame}

\begin{frame}
  \frametitle{GARD disease categories}
  \begin{center}
    \tiny
    \begin{tabular}{rl}\toprule
      Category & Count\\ \midrule
      Congenital and Genetic Diseases&3040\\
      Nervous System Diseases&1254\\
      Musculoskeletal Diseases&652\\
      Skin Diseases&591\\
      Eye diseases&573\\
      Rare Cancers&532\\
      Metabolic disorders&509\\
      Blood Diseases&321\\
      Kidney and Urinary Diseases&290\\
      Endocrine Diseases&263\\
      Digestive Diseases&248\\
      Ear, Nose, and Throat Diseases&242\\
      Mouth Diseases&210\\
      Heart Diseases&176\\
      Chromosome Disorders&151\\
      Immune System Diseases&148\\
      Lung Diseases&137\\
      Female Reproductive Diseases&89\\
      Newborn Screening&84\\
      Male Reproductive Diseases&70\\
      Bacterial infections&58\\
      Viral infections&39\\
      Parasitic diseases&33\\
      Hereditary Cancer Syndromes&26\\
      Connective tissue diseases&22\\
      Fungal infections&12\\
      Autoimmune / Autoinflammatory diseases&9\\
      Behavioral and mental disorders&7\\
      Nutritional diseases&3\\
      Environmental Diseases&2        \\ \bottomrule
    \end{tabular}
  \end{center}
\end{frame}

\begin{frame}
  \frametitle{Disease resources}
  \begin{center}
    \begin{tabular}{rl}
      GARD &
      \href{https://rarediseases.info.nih.gov/}{\texttt{https://rarediseases.info.nih.gov/}}\\
      Orphanet &
      \href{https://www.orpha.net}{\texttt{https://www.orpha.net}}\\
      GHR&
      \href{https://ghr.nlm.nih.gov/}{\texttt{https://ghr.nlm.nih.gov/}}\\
      Disease Ontology (DO) &
      \href{http://disease-ontology.org/}{\texttt{http://disease-ontology.org/}}\\
      OMIM & \href{https://omim.org/}{\texttt{https://omim.org/}}\\
      MeSH &
      \href{https://meshb.nlm.nih.gov}{\texttt{https://meshb.nlm.nih.gov}}\\
      MEDLINE+ &
      \href{https://medlineplus.gov/}{\texttt{https://medlineplus.gov/}}\\
      MONDO &
      \href{http://monarchinitiative.org/}{\texttt{http://monarchinitiative.org/}}\\
      HPO & \href{https://hpo.jax.org}{\texttt{https://hpo.jax.org}}\\
      NORD & \href{https://rarediseases.org}{\texttt{https://rarediseases.org}}\\
      MEDGEN & \href{https://www.ncbi.nlm.nih.gov/medgen/}{\texttt{https://www.ncbi.nlm.nih.gov/medgen/}}\\
      NCI & \href{https://ncit.nci.nih.gov/}{\texttt{https://ncit.nci.nih.gov/}}\\
      ICD10 & \href{https://www.icd10data.com/}{https://www.icd10data.com/}\\
      EFO & \href{https://www.ebi.ac.uk/efo/}{https://www.ebi.ac.uk/efo/}
    \end{tabular}
  \end{center}
\end{frame}


\begin{frame}
  \frametitle{Disease overlap matrix}
  \begin{block}{Direct match by synonyms or identifiers}
    \begin{center}\tiny
      \begin{tabular}{rcccccc}\toprule
        & GARD & Orphanet & GHR & DO & OMIM & MeSH\\ \midrule
        GARD & \textbf{6,504}&4,377&837&2,760&3,597&4,017\\
        Orphanet &4,299&\textbf{9,290}&616&3,301&4,427&3,613\\
        GHR &829&616&\textbf{1,287}&620&556&876\\
        DO &2,506&3,554&617&\textbf{8,699}&3,473&3,295\\
        OMIM &4,850&7,313&883&5,842&\textbf{12,883}&8,505\\
        MeSH &3,889&3,330&836&3,140&5,871&\textbf{8,892}\\
        MEDLINE+ &310&414&99&966&495&743\\
        MONDO &6,573&11,856&1,245&10,776&8,696&8,555\\
        HPO &730&868&143&1,510&1,458&1,382\\
        NORD & 1,049 & 617 &364 &741 &703 &937 \\
        MEDGEN &4,896 &7,753 &989 &6,796 &7,147 &9,400 \\
        NCI & 1,465 &1,435 &501 &2,319 &1,633 &2,430 \\
        ICD10 & 963 & 1,252& 167& 3,015& 1,258&2,394\\ \bottomrule
      \end{tabular}
    \end{center}
    The matrix is asymmetric due to $1-n$ and/or $m-1$ mappings; e.g.,
    there are $4,377$ GARD diseases that mapped to $4,299$ diseases in
    Orphanet. These numbers improve as we extend the
    matching indirectly to two-, three-, four-neighbor.
  \end{block}
\end{frame}

\begin{frame}
  \frametitle{Disease overlap matrix (cont'd)}
  \begin{center}\tiny
    \begin{tabular}{rccccccc}\toprule
      & MEDLINE+&MONDO & HPO & NORD & MEDGEN & NCI & ICD10\\ \midrule
      GARD & 257&5,796&715&1,153&4,075 & 1,410 &964\\
      Orphanet &280&8,973&858 & 642& 5,952& 1,439 &1,245\\
      GHR &99 &1,125 &143 &365 &879 &490 &187\\
      DO &722 &8,679 &1,502 &810 &4,417 &2,108 &2,525\\
      OMIM &664 &11,374 &2,491 &1,053 &9,010 &2,657 &2,029\\
      MeSH &653 &7,882 &1,153 &1,003 &7,357 &2,074 &1,896\\
      MEDLINE+ &\textbf{2,238} &662 &440 &202 &801 &765 &1,228\\
      MONDO &645 &\textbf{21,826} &2,385 &1,633 &11,672 &3,377 &3,273\\
      HPO &379 &2,237 &\textbf{13,725} &277 &2,160 &1,072 &1,695\\
      NORD &190 &1,163 &249 &\textbf{1,251} &938 &622&371\\
      MEDGEN &668 &13,221 &2,934 &1,361&\textbf{37,546} &3,632 &4,727\\
      NCI &593 &3,329 &1,092 &703 &3,450&\textbf{5,323} &1,880\\
      ICD10 &1,213 & 3,379& 1,820& 368& 4,301& 2,024&\textbf{16,066}\\ \bottomrule
    \end{tabular}
  \end{center}
\end{frame}

\begin{frame}
  \frametitle{So how many rare diseases are there?}
  According to MONDO,\footnote{\href{https://www.nature.com/articles/d41573-019-00180-y}{How many rare diseases are there?}} there are currently \textbf{10,393} rare
  diseases. This number includes diseases that are (i) not considered
  as rare by GARD (e.g., \emph{Klinefelter syndrome}) and (ii) having
  UMLS semantic type other than \emph{Disease or Syndrome} (e.g.,
  \emph{ovary leiomyosarcoma} is considered as \emph{Neoplastic
    Process}).
  \vskip1em
  New rare diseases are being added on a regular basis; e.g., a new
  entry is being considered for GARD at this moment:
  \begin{itemize}
  \item \emph{SLC6A1 Epileptic Encephalopathy}
  \item \emph{SLC6A1-related Disorders}
  \end{itemize}
  \vskip1em
  \emph{Stay tuned...}
\end{frame}

\begin{frame}
  \frametitle{Disease harmonization challenges}
  \begin{block}{How many ``distinct'' diseases are here?}
    \centerline{\includegraphics[width=9cm]{graph1}}
  \end{block}
\end{frame}

\begin{frame}
  \frametitle{A strategy for harmonization}
  \begin{enumerate}
  \item Generate \emph{strongly connected components}
    \begin{itemize}
    \item Perform transitive closure on nearest (weighted) neighbors
    \end{itemize}
  \item For each strongly connected component, do the following:
    \begin{enumerate}
    \item Calculate pair-wise similarities for all synonyms and
      identifiers based on Jaccard metric \[\text{sim}(x,y) =
      \frac{|x\cap y|}{|x\cup y|}\]
    \item Group synonyms and identifiers based on $\text{sim}(x,y) \ge
      \delta$ ($\delta=0.4$ gives reasonable grouping)
    \item Each such synonym/identifier grouping serves as an initial
      seed for which entities are then merged/split
    \end{enumerate}
  \end{enumerate}
\end{frame}

\begin{frame}
  \frametitle{Next steps}
  \begin{enumerate}
  \item Ongoing evaluation of the proposed harmonization strategy
  \item How best to normalize entities?
    \begin{itemize}
    \item Identify preferred synonyms and identifiers
    \end{itemize}
  \item Support manual curation and harmonization cycle
  \end{enumerate}
\end{frame}

\begin{frame}[fragile]
  \frametitle{Example cypher queries on disease knowlege graph}
  \begin{block}{Data source listing}
    \small
\begin{verbatim}
match(n:DATASOURCE) return n.name,n.instances
\end{verbatim}
  \end{block}

  \begin{block}{OMIM disease count}
    \small
\begin{verbatim}
match(n:`S_OMIM`:T047) return count(n)
\end{verbatim}
  \end{block}
  
  \begin{block}{GARD disease category breakdown}
    \small
\begin{verbatim}
match(n:`S_GARD`)-[]-(m:DATA) where m.is_rare=true 
with distinct labels(n) as t, count(n) as cnt 
unwind  t as Category
return Category,sum(cnt) as Count order by Count desc
\end{verbatim}
  \end{block}
\end{frame}

\begin{frame}[fragile]
  \frametitle{Example of overlap query}
  \begin{block}{Orphanet and MONDO}
    \small
\begin{verbatim}
match(n:`S_ORDO`)-[]-(m:DATA) 
where not exists(m.symbol) 
and not exists(m.reason_for_obsolescence) 
with n match(n)-[:N_Name|
   :I_CODE*1]-(a:`S_MONDO`)-[]-(b:DATA) 
where exists(b.label) and not exists(a.status) 
return count(distinct n) as `Orphanet`,
    count(distinct a) as `MONDO`
\end{verbatim}
  \end{block}
\end{frame}
\end{document}
