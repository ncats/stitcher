\PassOptionsToPackage{utf8}{inputenc}
\documentclass{bioinfo}
\copyrightyear{2020} \pubyear{2020}

\usepackage[colorlinks=true,urlcolor=black,citecolor=blue]{hyperref}
\usepackage[ruled,vlined]{algorithm2e}
\usepackage{multicol}

\access{Advance Access Publication Date: Day Month Year}
\appnotes{Manuscript Category}

\newcommand\st{\textbf{Stitcher}}
\newcommand\ix{\textbf{InXight Drugs}}
\newcommand\bangedup{\framebox{\parbox{\textwidth}{\it Dear Reviewer:
Due to technical difficulties with our typesetting software, we were
unable to generate proper references for this paper before the
submission deadline. However, the complete ``source code'' and
bibliographies of the paper are available
at \url{https://github.com/ncats/stitcher/paper} should you have any
concerns.}}}

\begin{document}
\firstpage{1}

\subtitle{Database and ontologies}

\title[Stitcher: An entity resolution framework]{Stitcher: An entity
resolution framework for comprehensive data integration of approved
drugs}
\author[Nguyen \textit{et~al}.]{Dac-Trung Nguyen,$^{\text{\sfb 1},\ast}$
Ivan Grishagin,$^{\text{\sfb 1}}$
Daniel Katzel,$^{\text{\sfb 1}}$
Tyler Peryea,$^{\text{\sfb 1,2}}$
Ajit Jadhav,$^{\text{\sfb 1}}$
and Noel Southall\,$^{\text{\sfb 1},\ast}$}
\address{$^{\text{\sf 1}}$Division of Pre-clinical Innovation,
National Center for Advancing Translational Sciences (NCATS), National
Institutes of Health, USA\\ 
$^{\text{\sf 2}}$Present address: Office of Health Informatics, Office
of Chief Scientist, Food and Drug Administration, USA} 

\corresp{$^\ast$To whom correspondence should be addressed.}

\history{Received on XXXXX; revised on XXXXX; accepted on XXXXX}

\editor{Associate Editor: XXXXXXX}

\abstract{\textbf{Motivation:}
As biomedical data continues to grow at a rapid rate, the need to
integrate diverse data across scales and modalities has never been
more urgent. In the context of drug data, the data integration
challenges are considerably more complex due to the lack of clarity
around the term ``drug.'' What is considered as ``drug'' can,
therefore, vary significantly between data sources. The data
integration challenges and the lack of a comprehensive resource of
drugs that been marketed or approved in the United States for human
use are the primary goals behind our current work.\\
\textbf{Results:}
Through the combination of a reference data source
(G-SRS) and entity resolution strategies tailored specifically for
drug entity type, we have developed a comprehensive data integration
pipeline (\st) for drug data. The
resource \url{https://drugs.ncats.io} is a comprehensive drug
resource that is supported by \st. \\
\textbf{Availability:} The complete source code along with data and
build instructions for \st{} is readily available on
Github \url{https://github.com/ncats/stitcher}. The latest version of
the stitch multigraph is available as a Neo4j database
at \url{https://stitcher.ncats.io/browser}
(use \texttt{stitcher.ncats.io:80} as the hostname). And the \ix{} resource is
accessible at \url{https://drugs.ncats.io}.\\ 
\textbf{Contact:} \href{mailto:southalln@mail.nih.gov}{southalln@mail.nih.gov}\\
\textbf{Supplementary information:} Supplementary data are available
at \textit{Bioinformatics} online.\\%[2em] 
%\bangedup
}

\maketitle

\section{Introduction}
As the volume of biological data continues to grow at an unprecedented
rate, data de-duplication---also commonly known as record linkage
or \emph{entity resolution}---is proportionally playing a prominent
role in data integration. From the construction of training data for
machine learning to building knowledge graphs as epistemological
frameworks for artificial intelligence, proper entity resolution is
essential in generating ground-truth data. The core challenge of
entity resolution is in establishing \emph{uniqueness}. For
well-defined entity types (e.g., gene, tissue, cell line), uniqueness
is determined solely based on established identifiers and
nomenclature; for other entity types (e.g., drug, disease, phenotype),
however, uniqueness is not as well-established due to conceptual
ambiguities in how entities are defined and represented. Take the
disease entity type as an example. The discrepancy between the
theoretical concept of ``disease entity'' from its clinical
nosology \citep{Hucklenbroich14} is what makes disease entity
resolution extremely challenging.

Herein we report on our recent data integration effort to build a
comprehensive resource of drugs that have either been marketed or
approved in the United States for human use. Such a resource is not
only instrumental for drug repurposing but also serves as a valuable
tool to further our understanding of the mechanistic properties of
molecular targets \citep{Huang2011,Huang2019}. To the best of our
knowledge, \ix{} is currently the most comprehensive resource of its
kind. In the remainder of this paper, we discuss data integration
challenges associated with drug data, conceptually as well as
technically. This discussion serves as the backdrop for the
development of \st, an entity resolution framework that we have
developed to address the shortcomings of traditional approaches. 

\subsection{What is a ``drug''?}
While the word is included within the name of the organization, the
U.S. Food and Drug Administration (FDA) does not have a
straightforward definition of the word ``drug.'' The Federal Food Drug
and Cosmetic Act (FD\&C Act) and FDA regulations define the term drug,
in part, by reference to its intended use, as ``articles intended for
use in the diagnosis, cure, mitigation, treatment, or prevention of
disease” and “articles (other than food) intended to affect the
structure or any function of the body of man or other
animals.'' \citep{FDADrug} More practically, the agency defines ``drug
substance'' and ``drug product'' respectively as the physical
ingredients found in marketed products. Others use the word ``drug''
to sometimes refer to ``drug substances'' and sometimes to ``drug
products'' as convenient, and this causes a great deal of semantic
confusion within drug data found on the web. The National Library of
Medicine produces a semantic product, RxNorm, that provides a variety
of precise semantic types for ingredients, tradenames, dose forms,
semantic clinical drug components, semantic clinical drug forms, and
semantic clinical drugs which facilitate working with drug data, but
its terminology is unfortunately limited to commonly used prescription
drugs, ``clinically significant ingredients,'' and adoption of this
complex semantic scheme is limited \citep{RxNorm}. 

There is a third definition of the word drug that is commonly used in
the literature and used by the FDA when it refers to an active moiety
and a new molecular entity. In this case, ingredients whose
pharmacological effect occurs through the same molecular entity are
considered the same drug. This holds for different salt forms such as
sumatriptan succinate and sumatriptan hemisulfate, but it also holds
for prodrugs and their metabolized active forms such as brincidofovir
and cidofovir \citep{NME}. \emph{An active moiety is a molecule or
ion, excluding those appended portions of the molecule that cause the
drug to be an ester, salt (including a salt with hydrogen or
coordination bonds), or other noncovalent derivative (such as a
complex, chelate, or clathrate) of the molecule, responsible for the
physiological or pharmacological action of the drug
substance} \citep{CFR2012}. Under the Food and Drug Administration
Amendments Act of 2007, all newly introduced active moieties must
first be reviewed by an advisory committee before the FDA can approve
these products.

As in other information domains, the names used to refer to drug
substances and products are particularly problematic because their
definitions change as a function of location or jurisdiction, time and
context. FDA and other national regulators of medicines have
collaborated to produce ISO 11238 \citep{ISO11238} which endeavors to
define an information scheme for the unambiguous identification of all
ingredients found in medicinal products, and FDA uses an
implementation of ISO 11238 as the backbone of its information systems
within the agency \citep{GSRS}. While this facilitates data exchange within
the FDA and with other national authorities, the task still remains to
be able to map other, external data sources into this
rigorously-defined scheme using whatever names and data are at hand. 

%\subsection{When are two drugs equivalent?}
%\begin{itemize}
%\item Layout the challenges in determining when two drugs are equivalent. This will depend on drug classes. For example, for small molecules, discuss salt forms, metals, and esters; for biologics, biosimilar; etc.
%\item Discuss the different types of identifier; INN, USAN, IUPAC, InChI, CAS, UNII, PubChem, company code, etc. Also address the challenge on the evolution of the drug identifier from discovery (where stereochemistry can be ambiguous) to approval.
%\end{itemize}

%\enlargethispage{12pt}

\section{Approach}

\subsection{Preliminary concepts}
The conceptual data model underlying \st{} is a \emph{multigraph}.
Within this multigraph, a node can either be a \emph{stitch node}
or \emph{data node}. Each data node represents a ``raw'' entity as
ingested from the data source; its corresponding stitch node is
a \emph{standardized} representation that is used
for \emph{stitching}. An edge between two stitch nodes can either be
a \emph{stitch key} (undirected) or \emph{relationship} (directed). A
unique \emph{stitch value} is associated with each stitch key such
that it forms a clique. Figure~\ref{fig:graph1} shows an instance of a
connected component of a stitch multigraph with overlapping cliques. 

A connected component in the stitch multigraph represents the basic
unit of work for entity resolution. While the majority of connected
components are of reasonable sizes (e.g., 20 to 50 stitch nodes), the
real challenges center around effective strategies for handling very
large connected components---or also commonly known
as \emph{hairballs} \citep{Croset2015}. For example, the current
version of the \ix{} resource has an hairball close to 30,000 stitch
nodes spanning across 15 data sources. We discuss our strategies in
detail for untangling through such an hairball in
Section~\ref{sec:methods-er}. 

The primary goal of entity resolution is to determine the number of
unique entities in a connected component. These derived entities are
represented as \emph{sgroup nodes} in the stitch multigraph. There can
be multiple instances of sgroup nodes for any given set of stitch
nodes, with each instance reflects a specific algorithmic strategy or
version. Figure~\ref{fig:graph1} shows that there is only one unique
entity as determined by the entity resolution algorithm for the given
connected component. 

\begin{figure}[!tpb]
\centerline{\includegraphics[scale=0.5]{graph3}}
\caption{A connected component in the stitch multigraph with
four \emph{stitch nodes} (medium) and corresponding \emph{data nodes}
(small). Each stitch value forms a clique within this connected
component. The edge labels between stitch nodes are the stitch keys.
The large node is the derived entity (i.e., sgroup node) generated
from entity resolution.}\label{fig:graph1} 
\end{figure}

\subsection{Stitch keys}\label{sec:stitch-keys}
Stitch key is a core concept in \st. It defines how entities are
matched, which, in turn, determines how cliques and connected
components are formed. By virtue of its importance, the stitch key
should reflect the true identity of the entity as much as possible.
Depending on the entity type, the stitch key can be generic (e.g.,
synonym) or very specific (e.g., molecular hash key). For drug entity
type, \st\ relies on the following stitch keys for each entity: 
\begin{unlist}
\item{\texttt{N\_Name}.} This is the most generic stitch key
available. Stitch values associated with this stitch key can be any
established names or nomenclature; e.g., tradenames, INN
(International Nonproprietary Names), USAN (United States Adopted
Names), IUPAC (International Union of Pure and Applied Chemistry). 
\item{\texttt{I\_UNII}, \texttt{I\_CAS}, \texttt{I\_CID}, \texttt{I\_CODE}.}
These stitch keys represent (i) unique identifiers assigned to the entity by
a well-known registrar (e.g., the U.S. Food and Drug Administration in
the case of UNII) or (ii) internal company code. \texttt{I\_UNII},
\texttt{I\_CAS}, and \texttt{I\_CID} are specific to drug (or
substance in general) entity type, whereas \texttt{I\_CODE} can be
used for any type of identifiers. The decision to use specific stitch
keys over generic ones ultimately rests on the strategies used for
entity resolution.
\item{\texttt{H\_LyChI\_L5}, \texttt{H\_LyChI\_L4}, \texttt{H\_LyChI\_L3}.}
For the small molecule class of drugs, perhaps more important than any
identifiers is the underlying chemical structure definition. These
stitch keys are hash values derived from the molecular structure at
different resolutions \citep{lychi}. Section~\ref{sec:methods-ingest}
discusses in detail how these derived stitch values are generated. 
\item{\texttt{R\_activeMoiety}.} Technically not a stitch key, the
active moiety relationship between two drugs provides a strong
evidence of equivalence. While this relationship can be inferred
directly from the chemical structures (e.g., freebase and salt forms,
with and without esters), there is some level of curation needed to
handle structures with metal complex. 
\end{unlist}
Table~\ref{tab:imatinib} shows an example of stitch keys and stitch values for
the drug entity \emph{imatinib mesylate}. In this example,
the \texttt{R\_activeMoiety} relationship specifies the UNII of the
freebase form of imatinib mesylate. 

\begin{table}[thb]
\processtable{Stitch keys and stitch values for the
drug \emph{imatinib mesylate}\label{tab:imatinib}}
{\begin{tabular}{@{}ll@{}}\toprule Stitch key &
Stitch value\\\midrule
\texttt{N\_Name} & \texttt{IMATINIB MESYLATE}; \texttt{GLEEVEC}; \texttt{GLIVEC}\\
\texttt{I\_UNII} & \texttt{8A1O1M485B}\\
\texttt{I\_CAS} & \texttt{220127-57-1}\\
\texttt{I\_CID} & \texttt{5291}\\
\texttt{I\_CODE} & \texttt{STI-571}; \texttt{CHEMBL941}\\
\texttt{H\_LYCHI\_L5} & \texttt{7S4GKGNQ6N3X-N}\\
\texttt{H\_LYCHI\_L4} & \texttt{VLU17BQBSGWU-N}; \texttt{K83X3L3XSSHK-S}\\
\texttt{H\_LYCHI\_L3} & \texttt{VL3FPUQ59CU-N}; \texttt{K846NBMB7T3-S}\\
\texttt{R\_activeMoiety} & \texttt{BKJ8M8G5HI}\\\botrule
\end{tabular}}{}
\end{table}

\subsection{Data sources}
\st\ utilizes a number of diverse data sources for the \ix{} resource.
Among the data sources, of particular importance is the public G-SRS
data source from the FDA \citep{GSRSData}. This data source is
well-curated and contains over 100K substances across six different
classes: chemical, structurally diverse, protein, mixture, polymer,
and nucleic acid. As a data source derived from the FDA's internal
substance registry system \citep{GSRS}, the G-SRS data source
naturally forms the basis of our data integration effort. Using this
data source as the ``seed'' from which other data sources can map onto
has the following benefits: 
\begin{itemize}
\item Since the G-SRS data source implements the ISO 11238
standard \citep{ISO11238} for defining medicinal substances, it serves
as an ideal starting point for what constitutes a ``drug.''
\item The data is a public version of the internal substance registry
within the FDA; as such, it is well-curated and up-to-date.
\end{itemize}
The complete list of data sources currently used by \st\ is shown in
Table~\ref{tab:data-sources}. 

\begin{table}[thb]
\processtable{Data sources used in the current version of \st.\label{tab:data-sources}}
{\begin{tabular}{@{}ll@{}}\toprule
Data source & Size\\ \midrule
G-SRS, April 2019&	105,019\\
Withdrawn and Shortage Drugs List Feb 2018 &	674\\
Broad Institute Drug List 2018-09-07 &	6,125\\
NCATS Pharmaceutical Collection, April 2012 &	14,814\\
Rancho BioSciences, March 2019 &	51,591\\
Pharmaceutical Manufacturing Encyclopedia (Third Edition) &	2,268\\
DailyMed Rx, January 2019 &	74,850\\
DrugBank, December 2018&	11,922\\
DailyMed Other, January 2019&	13,393\\
DailyMed OTC, January 2019&	79,448\\
Drugs\@FDA \& Orange Book, July 2019&	28,256\\
ClinicalTrials, December 2017&	305,833\\
OTC Monographs, December 2018&	2,713\\
FDA NADA and ANADAs, December 2018&	554\\
FDA Excipients, December 2018&	10,212\\ \botrule
\end{tabular}}{}
\end{table}


\subsection{Overall strategy}
The basic premise behind \st{} is that data integration is often done
within the context of a specific data source. This is a reasonable
assumption given the data quality varies when integrating across
disparate sources. Furthermore, by establishing a reference data source
for data integration, we have finer control over the following: 
\begin{unlist}
\item{\emph{Data quality}.} A reference data source is typically
selected such that it is of high quality. Here, we can also impose
other data quality constraints (e.g., no synonyms can span multiple
entities) to guide entity resolution. 
\item{\emph{Data resolution}.} Entity resolution is particularly
challenging when data integration involves ontologies. A reference
data source can serve as the anchor ontology from which other
ontologies can be mapped. As with data quality, we can also impose any
additional semantic constraints; e.g., prostate cancer is not one of
the diagnoses for a female patient in an electronic health record. 
\item{\emph{Data curation}.} Generating ground-truth data is more
manageable with a single data source than across multiple data sources.
This is particularly important due to the iterative feedback between
data curation and data integration. 
\end{unlist}
The G-SRS data source serves as an ideal reference data source. Its
rigorous substance models and well-structured data elements give us a
good starting point for drug data integration. In the next section, we
discuss our strategies in utilizing the G-SRS reference data source to
address entity resolution for drug data. 

\begin{methods}
\section{Methods}\label{sec:methods}
In general, data integration with \st{} consists of four basic
steps: \emph{ingestion}, \emph{stitching}, \emph{entity resolution},
and \emph{entity normalization}. With the exception of \emph{entity
resolution}, all other steps---as they are currently implemented
in \st---are generic and can be applied to a wide range of entity
types. 

\subsection{Data ingestion}\label{sec:methods-ingest}
\st{} is capable of ingesting data in a wide variety of sources and
formats. Semantic formats such as OWL, RDF, and Turtle are supported
as are JSON, delimiter separated text, and custom formats. For
non-semantic format, a separate configuration file is required to map
properties to stitch keys. 

An important step in data ingestion is the standardization and
validation of stitch values. For \texttt{N\_Name} stitch key, the
standardization procedure is simply to convert the input string to
uppercase; no validation is performed. For \texttt{I\_UNII}
and \texttt{I\_CAS} stitch keys, no standardization is required, and
validation is a simple checksum calculation to ensure the stitch value
is proper. Depending on the input format, \st\ also provides basic
utilities (e.g., regular expression) to help with data transformation
during ingestion. 

Perhaps the most unique feature of \st\ is its ability to incorporate
knowledge of chemical structures into entity resolution. Whereas
traditional approaches rely on names and identifiers to determine
equivalence substances, \st\ goes a step further and utilizes the
underlying chemical structures to infer equivalence. This 
is particularly relevant when the drug is a mixture, prodrug, or
active moiety with complex excipient (or derivative thereof). As an
example, consider the drug entity \emph{IMATINIB MESYLATE} and its
active ingredient \emph{IMATINIB}. Here, it is obvious that the two
entities cannot be matched by name alone. Instead, having structural
information by way of molecular hash keys for each molecular component
allows us to determine equivalence from the common active
moiety \emph{IMATINIB} between the two entities. This trivial example
might suggest that, instead of comparing names exactly, we find the
longest common substring of the names. The approach would certainly
work in this example, but to make it work in general would require
very specialized parsing rules and dictionaries. 

For data sources with chemical structures, the most computationally
demanding step in data ingestion is the generation of molecular hash
keys. Hash keys are generated for each component of a chemical
structure in three different structural
levels: \texttt{L5}, \texttt{L4}, and \texttt{L3}, which correspond to
stitch keys \texttt{H\_LYCHI\_L5}, \texttt{H\_LYCHI\_L4},
and \texttt{H\_LYCHI\_L3}, respectively. Level \texttt{L5} is the most
specific; it represents the chemical structure as-is, i.e., without
structure normalization and standardization. With the exception of the
relation \texttt{R\_activeMoiety}, a match at this level has higher
priority over other stitch keys. The next level \texttt{L4} represents
the structure after normalization and standardization per the LyChI
software package \citep{lychi}. A match at this level implies that two
structures are equivalent 
%insofar as \textcolor{red}{the valence bond theory is valid}
in terms of stereochemistry, resonance, and tautomerism. And the last
level \texttt{L3} is the same as \texttt{L4} but without
stereochemistry. A match at this level is considered weak and does not
constitute equivalence without other significant supporting evidence.
The purpose for \texttt{L3} is in anticipation of incorrect or missing
stereo information, which is one of the most common type of errors
associated with chemical structures. For each hash key, a
suffix \texttt{-M}, \texttt{-S}, or \texttt{-N} is also assigned to
designate the molecular component as either
a \emph{metal}, \emph{salt}, or \emph{neither}, respectively.
Table~\ref{tab:imatinib} illustrates all three representations for the
drug \emph{IMATINIB MESYLATE}. Note that the cardinality
for \texttt{L5} is always one, whereas for \texttt{L4} and \texttt{L3}
the cardinality is equal to the number of non-hydrate molecular
components. (Hydrate components are removed prior to processing.) 

\subsection{Data stitching}
\emph{Stitching} is the process by which the stitch multigraph is
incrementally constructed as data is ingested.
Algorithm~\ref{algo:stitching} describes the basic stitching algorithm
of \st. This algorithm is applied to each data source, and upon its
completion produced a stitch multigraph such that any stitch value
that spans $N$ stitch nodes is an induced clique, i.e., a complete
subgraph of $N$ nodes and $\frac{N(N-1)}{2}$ edges. Overlapping
induced cliques form the basis for the proposed entity resolution
approach discussed in the next section. As a side-effect, the
stitching algorithm also utilizes the union-find
algorithm \citep{Cormen2001} to efficiently track connected
components. 

\begin{algorithm}\label{algo:stitching}
\SetKwInOut{Input}{Input}
\SetKwInOut{Output}{Output}
\SetKwFunction{Union}{Union}
\SetKwFunction{Find}{Find}
\SetAlgoLined
\DontPrintSemicolon
Let $W$ denote the set of stitch nodes created in the data ingestion
step for a given data source $D$.\\
Let $\langle k, v\rangle$ be the tuple of stitch key and value,
respectively, defined for a stitch node $w$.\\
$G=(V,E)$ is the current stitch multigraph.\\
\Find{$k, v$} is a function that returns all stitch nodes in $V$
containing stitch key $k$ and stitch value $v$.\\
\Union{$w,z$} is a union-find algorithm for tracking disjoint sets
(i.e., connected components).\\
 \For{$w \in W$}{
    \For{$\langle k_i, v_i\rangle \in w$}{
      \For{$z \in$ \Find{$k_i,v_i$}}{
         $E \leftarrow E \cup z\thicksim w$\;
         \Union{$w,z$}\;
      }
    }
    $V \leftarrow V \cup w$\;
 }
 \caption{Entity stitching algorithm}
\end{algorithm}

\subsection{Entity resolution}\label{sec:methods-er}
After all data sources have been stitched together, the next step is
to identify \emph{unique} entities from the stitch multigraph. Formally, this
step is known as \emph{entity resolution} and is the only step
within \st\ that is specific to the drug entity type. (This is to be
expected: Given that entity resolution is about adjudicating
the splitting and merging of entities, a reasonable amount of
knowledge of the entity type is required for the adjudication to be
effective.) For a given connected component, the iterative process of
assigning equivalence labels to stitch nodes is known
as \emph{untangling}. Algorithm~\ref{algo:untangle} gives a high level
outline of the untangling process.

\begin{algorithm}\label{algo:untangle}
\SetKwFunction{MergeNodes}{MergeNodes}
\SetKwFunction{MergeCliques}{MergeCliques}
\SetKwFunction{MergeSingletons}{MergeSingletons}
\SetAlgoLined
\DontPrintSemicolon
Let $U$ be the disjoint set data structure for all entities.\\
Let $S$ denote the set of unlabeled entities (i.e., singletons).\\
$C=(V,E)$ is the connected component.\\
\MergeNodes{$U, r$} is a function that performs transitive closure on
stitch nodes in $C$ which are connected by a relation $r\in E$. The results are
accumulated in $U$.\\
\MergeCliques{$U, K$} is a function that takes a set of stitch keys
$K$, finds overlapping cliques that span two or more stitch keys,
and performs transitive closure on the entities.\\
\MergeSingletons{$U, S, K$} is a function that also takes in a set of
stitch keys $K$, a set of singleton stitch node $S$, and find the best
mapping to an already labeled stitch node.\\
\MergeNodes{$U,$\texttt{R\_activeMoiety}}\;
\MergeNodes{$U,$\texttt{I\_UNII}}\;
\MergeNodes{$U,$\texttt{H\_LyChI\_L4}}\;
\MergeCliques{$U,$\texttt{N\_Name},\texttt{I\_CAS},\texttt{I\_CID},\texttt{H\_LyChI\_L4}}\;
\MergeSingletons{$U,S,$\texttt{N\_Name},\texttt{I\_CAS}}\;
 \caption{An algorithm to untangle a connected component}
\end{algorithm}
At the core of the algorithm is the implied priorities associated with
the stitch keys. The relation \texttt{R\_activeMoiety} has the highest
priority as it is manually generated and is a special relation
that only available to the G-SRS data source. As an
example, consider the entities \emph{acetylsalicylic acid}
(or also commonly known as \emph{aspirin}) and \emph{ethyl acetylsalicylate}
shown in Figures~\ref{fig:aspirin} and \ref{fig:ethyl},
respectively. While the two entities have nothing in common, in G-SRS
\emph{acetylsalicylic acid} is annotated as being an \emph{active
moiety} of \emph{ethyl acetylsalicylate}. Further examination of the
structural differences shows that only an \emph{ester} separates the
two entities; this falls well within what the FDA considers as
equivalent drugs. This example also highlights a quagmire:
Computationally, there is nothing to prevent us from
imputing \emph{active moiety} relationships through efficient (sub-)
graph isomorphisms. This, however, is a very tempted trap that we have
thus far resisted due to other forms of \emph{active moiety} relationships---e.g.,
metabolites and metals---that would require considerable investment of effort.
\begin{figure}[!tpb]
\centering
\begin{minipage}{.45\textwidth}
\centerline{\includegraphics[scale=0.5]{aspirin-crop}}
\caption{Chemical structure for acetylsalicylic
acid.}\label{fig:aspirin}
\end{minipage}
\hfill
\begin{minipage}{.45\textwidth}
\centerline{\includegraphics[scale=0.5]{VX19C5613T-crop}}
\caption{Chemical structure for ethyl acetylsalicylate.}\label{fig:ethyl}
\end{minipage}
\end{figure}

The next priority is the stitch key \texttt{I\_UNII}. As UNII is
the primary identifier for the G-SRS data source, any data source that
provides mapping based on this identifier implies that the data source has
sufficient knowledge of G-SRS (i.e., guilt by association). For
entities that can be represented by chemical structures, the stitch
key \texttt{H\_LyChI\_L4} has the next level of priority. The
complexity required for two entities to have the same stitch values
means that the entities are less likely to match by errors. The rest
of the stitch keys
(i.e., \texttt{N\_Name}, \texttt{I\_CAS}, \texttt{I\_CID}) all have the
lowest priority.

At the completion of Algorithm~\ref{algo:untangle}, the disjoint set
data structure $U$ contains all equivalence entity classes such that
each class is represented by an \emph{sgroup} node in the stitch
multigraph. The sgroup nodes are the \emph{resolved entities}.

\subsection{Entity normalization}
The last step in the data integration pipeline is to decide how the
resolved entities are defined. This step is referred to as \emph{entity
normalization} and its goals are to have (i) clear and consistent
strategies for merging properties and (ii) conflict resolution
(semantic as well as self consistency). While this step can be quite
trivial if the properties are mutually exclusive across all data
sources, to address this in a general setting will require considerable
efforts in terms of understanding the data source and its metadata. Within the
context of the current work, we resort to a simple strategy: When
merging properties, we preferentially use those that come from G-SRS
with a basic consistency constraint that a synonym is never associated
with more than one resolved entity. It also helps that many of the
data sources in Table~\ref{tab:data-sources} have mutually exclusive
properties; e.g., the property ``drug approval year'' in the Drugs@FDA
data source is not a property of G-SRS.

%\enlargethispage{6pt}
\end{methods}

\section{Results}
\st\ and the data integration pipeline developed for the \ix\ resource
are available in source form
at \url{https://github.com/ncats/stitcher}. The stitch multigraph
built with data sources listed in Table~\ref{tab:data-sources} is also
available as a Neo4j database
at \url{https://stitcher.ncats.io/browser}. (While no credentials are
needed for the database, the web interface requires that the
string \texttt{stitcher.ncats.io:80} is entered into the
field \texttt{Host}.) This database currently contains 192,413 stitch
nodes and 11,948,470 edges (relationships and stitch
keys). Tables~\ref{tab:stitch-keys} and \ref{tab:stitch-values} give a
breakdown of the stitch- keys and values, respectively, in the stitch
multigraph. All figures used throughout this paper have been generated
directly from this database.

\begin{table}[thb]
\processtable{Distribution of edge types for the stitch multigraph.\label{tab:stitch-keys}}
{\begin{tabular}{@{}ll@{}}\toprule
Stitch key & Size\\ \midrule
\texttt{H\_LyChI\_L3}	& 6,140,942\\
\texttt{H\_LyChI\_L4} &	5,300,078\\
\texttt{N\_Name} &	177,446\\
\texttt{H\_LyChI\_L5} &	162,160\\
\texttt{I\_UNII} &	139,176\\
\texttt{R\_activeMoiety} & 14,940\\
\texttt{I\_CAS}	& 11,684\\
\texttt{I\_CID} &	2,044\\ \botrule
\end{tabular}}{}
\end{table}

\begin{table}[thb]
\processtable{Top stitch value for each stitch key.\label{tab:stitch-values}}
{\begin{tabular}{@{}lll@{}}\toprule
Stitch key & Stitch value & Size\\ \midrule
\texttt{H\_LyChI\_L3}	& \texttt{VUSPQLGXN18-M} &	1,344,440\\
\texttt{H\_LyChI\_L4} &	\texttt{VU8BQZFPPYTZ-M} & 1,307,592\\
\texttt{N\_Name} & \texttt{ROFECOXIB} &72 \\
\texttt{H\_LyChI\_L5} &	\texttt{9DKQLD7D29DN-N} & 162,160\\
\texttt{I\_UNII} & \texttt{UNKNOWN} &210 \\
\texttt{R\_activeMoiety} & \texttt{2M83C4R6ZB} & 106\\
\texttt{I\_CAS}	& \texttt{25322-68-3} & 1,806\\
\texttt{I\_CID} & \texttt{121225712} & 380\\ \botrule
\end{tabular}}{The LyChI hash keys \texttt{L3} and \texttt{L4}
correspond to the potassium ion (K+).}
\end{table}

In the remainder of this section, we provide examples that highlight
different entity resolution challenges.
\subsection{ASPIRIN}
\emph{ASPIRIN} is a versatile drug that can be used alone or in
combination with other drugs. Shown in Figure~\ref{fig:ASPIRIN} is the
induced subgraph of the much larger \emph{ASPIRIN} connected component that
forms the \emph{ASPIRIN} entity.
\begin{figure}[!tpb]
\centerline{\includegraphics[scale=0.5]{graph5-crop}}
\caption{A connected component for ASPIRIN.}\label{fig:ASPIRIN}
\end{figure}
This example demonstrates \st's ability to tease out only the relevant
stitch nodes for which \emph{ASPIRIN} is likely to be the active
moiety for the underlying substance.

\subsection{LEVOMETHADYL}
\emph{LEVOMETHADYL} and its derivative \emph{LEVACETYLMETHADOL} are
often considered as two separate drugs. This is readily apparent in
Figure~\ref{fig:LEVOMETHADYL}, which shows that there are two distinct
``clusters'' in the stitch multigraph. If entity resolution is based on
graph metrics (e.g., betweenness centrality), it is likely that this
connected component will yield two drugs instead of one. Here, the
priority of the stitch key allows the two clusters to be merged to
indicate that there is only one drug.
\begin{figure}[!tpb]
\centerline{\includegraphics[scale=0.5]{graph6-crop}}
\caption{A connected component for LEVOMETHADYL that clearly shows two
distinct clusters.}\label{fig:LEVOMETHADYL}
\end{figure}

\subsection{BENOXAPROFEN} Figure~\ref{fig:BENOXAPROFEN} shows the
connected component for \emph{BENOXAPROFEN}, a nonsteroidal
antiinflammatory drug approved in 1982. This drug is a racemic
mixture. The density of this connected component is a reflection of
the lack of specified stereocenter that caused many spurious stitch
keys. \st\ is able to disambiguate the connected component into three
distinct entities that represent the mixture, \emph{R-}, and \emph{S-}.
\begin{figure}[!tpb]
\centerline{\includegraphics[scale=0.5]{graph4-crop}}
\caption{A dense connected component for BENOXAPROFEN that resolved to
three unique entities.}\label{fig:BENOXAPROFEN}
\end{figure}

%\section{Discussion}

\section*{Acknowledgments}
We thank our colleagues, Mark Williams and Tyler Beck, for their
valuable proof-reading of early drafts of the manuscript. We also
thank our colleague, Tongan Zhao, for his help in developing a
prototype curation user interface for \st. We are particularly grateful
to Alexey Zakharov and Tim Sheils for their constant encouragement and
support.

\bibliographystyle{natbib}
%\bibliographystyle{achemnat}
%\bibliographystyle{plainnat}
%\bibliographystyle{abbrv}
%\bibliographystyle{bioinformatics}
%
%\bibliographystyle{plain}
%
\bibliography{stitcher}

\end{document}
