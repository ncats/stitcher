\PassOptionsToPackage{utf8}{inputenc}
\documentclass{bioinfo}
\copyrightyear{2020} \pubyear{2020}

\usepackage[colorlinks=true,urlcolor=black,citecolor=blue]{hyperref}
\usepackage[ruled,vlined]{algorithm2e}
\usepackage{multicol}

\access{Advance Access Publication Date: Day Month Year}
\appnotes{Manuscript Category}

\newcommand\st{\textbf{Stitcher}}
\newcommand\ix{\textbf{InXight Drugs}}
\newcommand\bangedup{\framebox{\parbox{\textwidth}{\it Dear Reviewer:
Due to technical difficulties with our typesetting software, we were
unable to generate proper references for this paper before the
submission deadline. However, the complete ``source code'' and
bibliographies of the paper are available
at \url{https://github.com/ncats/stitcher/paper} should you have any
concerns.}}}

\begin{document}
\firstpage{1}

\subtitle{Database and ontologies}

\title[Stitcher: An entity resolution framework]{Stitcher: An entity
resolution framework for comprehensive data integration of approved
drugs}
\author[Nguyen \textit{et~al}.]{Dac-Trung Nguyen,$^{\text{\sfb 1}}$
Ivan Grishagin,$^{\text{\sfb 1}}$
Daniel Katzel,$^{\text{\sfb 1}}$
Tyler Peryea,$^{\text{\sfb 1,2}}$
Ajit Jadhav,$^{\text{\sfb 1}}$
and Noel Southall\,$^{\text{\sfb 1},\ast}$}
\address{$^{\text{\sf 1}}$Division of Pre-clinical Innovation,
National Center for Advancing Translational Sciences (NCATS), National
Institutes of Health, USA\\ 
$^{\text{\sf 2}}$Present address: Office of Health Informatics, Office
of Chief Scientist, Food and Drug Administration, USA} 

\corresp{$^\ast$To whom correspondence should be addressed.}

\history{Received on XXXXX; revised on XXXXX; accepted on XXXXX}

\editor{Associate Editor: XXXXXXX}

\abstract{\textbf{Motivation:}
\\
\textbf{Results:} Text  Text Text Text Text Text Text Text Text Text  Text Text Text Text Text
Text Text Text Text Text Text Text Text Text Text Text Text Text  Text Text Text Text Text Text\\
\textbf{Availability:} The complete source code along with data and
build instructions for \st{} is readily available on
Github \url{https://github.com/ncats/stitcher}. The \ix{} resource is
accessible at \url{https://drugs.ncats.io}.\\ 
\textbf{Contact:} \href{mailto:southalln@mail.nih.gov}{southalln@mail.nih.gov}\\
\textbf{Supplementary information:} Supplementary data are available
at \textit{Bioinformatics} online.\\%[2em] 
%\bangedup
}

\maketitle

\section{Introduction}
As the volume of biological data continues to grow at an unprecedented
rate, data de-duplication---also commonly known as record linkage
or \emph{entity resolution}---is proportionally playing a prominent
role in data integration. From the construction of training data for
machine learning to building knowledge graphs as epistemological
frameworks for artificial intelligence, proper entity resolution is
essential in generating ground-truth data. The core challenge of
entity resolution is in establishing \emph{uniqueness}. For
well-defined entity types (e.g., gene, tissue, cell line), uniqueness
is determined solely based on established identifiers and
nomenclature; for other entity types (e.g., drug, disease, phenotype),
however, uniqueness is not as well-established due to conceptual
ambiguities in how entities are defined and represented. Take the
disease entity type as an example. The discrepancy between the
theoretical concept of ``disease entity'' from its clinical
nosology \citep{Hucklenbroich14} is what makes disease entity
resolution extremely challenging.

Herein we report on our recent data integration effort to build a
comprehensive resource of drugs that have either been marketed or
approved in the United States for human use. Such a resource is not
only instrumental for drug repurposing but also serves as a valuable
tool to further our understanding of the mechanistic properties of
molecular targets \citep{Huang2019}. To the best of our
knowledge, \ix{} is currently the most comprehensive resource of its
kind. In the remainder of this paper, we discuss data integration
challenges associated with drug data, conceptually as well as
technically. This discussion serves as the backdrop for the
development of \st, an entity resolution framework that we have
developed to address the shortcomings of traditional approaches. 

\subsection{What is a ``drug''?}
While the word is included within the name of the organization, the
U.S. Food and Drug Administration (FDA) does not have a
straightforward definition of the word ``drug.'' The Federal Food Drug
and Cosmetic Act (FD\&C Act) and FDA regulations define the term drug,
in part, by reference to its intended use, as ``articles intended for
use in the diagnosis, cure, mitigation, treatment, or prevention of
disease” and “articles (other than food) intended to affect the
structure or any function of the body of man or other
animals.'' \citep{FDADrug} More practically, the agency defines ``drug
substance'' and ``drug product'' respectively as the physical
ingredients found in marketed products. Others use the word ``drug''
to sometimes refer to ``drug substances'' and sometimes to ``drug
products'' as convenient, and this causes a great deal of semantic
confusion within drug data found on the web. The National Library of
Medicine produces a semantic product, RxNorm, that provides a variety
of precise semantic types for ingredients, tradenames, dose forms,
semantic clinical drug components, semantic clinical drug forms, and
semantic clinical drugs which facilitate working with drug data, but
its terminology is unfortunately limited to commonly used prescription
drugs, ``clinically significant ingredients,'' and adoption of this
complex semantic scheme is limited \citep{RxNorm}. 

There is a third definition of the word drug that is commonly used in
the literature and used by the FDA when it refers to an active moiety
and a new molecular entity. In this case, ingredients whose
pharmacological effect occurs through the same molecular entity are
considered the same drug. This holds for different salt forms such as
sumatriptan succinate and sumatriptan hemisulfate, but it also holds
for prodrugs and their metabolized active forms such as brincidofovir
and cidofovir \citep{NME}. \emph{An active moiety is a molecule or
ion, excluding those appended portions of the molecule that cause the
drug to be an ester, salt (including a salt with hydrogen or
coordination bonds), or other noncovalent derivative (such as a
complex, chelate, or clathrate) of the molecule, responsible for the
physiological or pharmacological action of the drug
substance} \citep{CFR2012}. Under the Food and Drug Administration
Amendments Act of 2007, all newly introduced active moieties must
first be reviewed by an advisory committee before the FDA can approve
these products.

As in other information domains, the names used to refer to drug
substances and products are particularly problematic because their
definitions change as a function of location or jurisdiction, time and
context. FDA and other national regulators of medicines have
collaborated to produce ISO 11238 [ref] which endeavors to define an
information scheme for the unambiguous identification of all
ingredients found in medicinal products, and FDA uses an
implementation of ISO 11238 as the backbone of its information systems
within the agency. [GSRS] While this facilitates data exchange within
the FDA and with other national authorities, the task still remains to
be able to map other, external data sources into this
rigorously-defined scheme using whatever names and data are at hand. 

\subsection{When are two drugs equivalent?}
%\begin{itemize}
%\item Layout the challenges in determining when two drugs are equivalent. This will depend on drug classes. For example, for small molecules, discuss salt forms, metals, and esters; for biologics, biosimilar; etc.
%\item Discuss the different types of identifier; INN, USAN, IUPAC, InChI, CAS, UNII, PubChem, company code, etc. Also address the challenge on the evolution of the drug identifier from discovery (where stereochemistry can be ambiguous) to approval.
%\end{itemize}

%\enlargethispage{12pt}

\section{Approach}

\subsection{Preliminary concepts}
The conceptual data model underlying \st{} is a \emph{multigraph}.
Within this multigraph, a node can either be a \emph{stitch node}
or \emph{data node}. Each data node represents a ``raw'' entity as
ingested from the data source; its corresponding stitch node is
a \emph{standardized} representation that is used
for \emph{stitching}. An edge between two stitch nodes can either be
a \emph{stitch key} (undirected) or \emph{relationship} (directed). A
unique \emph{stitch value} is associated with each stitch key such
that it forms a clique. Figure~\ref{fig:graph1} shows an instance of a
connected component of a stitch multigraph with overlapping cliques. 

A connected component in the stitch multigraph represents the basic
unit of work for entity resolution. While the majority of connected
components are of reasonable sizes (e.g., 20 to 50 stitch nodes), the
real challenges center around effective strategies for handling very
large connected components---or also commonly known
as \emph{hairballs} \citep{Croset2015}. For example, the current
version of the \ix{} resource has an hairball close to 30,000 stitch
nodes spanning across 15 data sources. We discuss our strategies in
detail for untangling through such an hairball in
Section~\ref{sec:methods-er}. 

The primary goal of entity resolution is to determine the number of
unique entities in a connected component. These derived entities are
represented as \emph{sgroup nodes} in the stitch multigraph. There can
be multiple instances of sgroup nodes for any given set of stitch
nodes, with each instance reflects a specific algorithmic strategy or
version. Figure~\ref{fig:graph1} shows that there is only one unique
entity as determined by the entity resolution algorithm for the given
connected component. 

\begin{figure}[!tpb]
\centerline{\includegraphics[scale=0.5]{graph3}}
\caption{A connected component in the stitch multigraph with
four \emph{stitch nodes} (medium) and corresponding \emph{data nodes}
(small). Each stitch value forms a clique within this connected
component. The edge labels between stitch nodes are the stitch keys.
The large node is the derived entity (i.e., sgroup node) generated
from entity resolution.}\label{fig:graph1} 
\end{figure}

\subsection{Stitch keys}\label{sec:stitch-keys}
Stitch key is a core concept in \st. It defines how entities are
matched, which, in turn, determines how cliques and connected
components are formed. By virtue of its importance, the stitch key
should reflect the true identity of the entity as much as possible.
Depending on the entity type, the stitch key can be generic (e.g.,
synonym) or very specific (e.g., molecular hash key). For drug entity
type, \st\ relies on the following stitch keys for each entity: 
\begin{unlist}
\item{\texttt{N\_Name}.} This is the most generic stitch key
available. Stitch values associated with this stitch key can be any
established names or nomenclature; e.g., tradenames, INN
(International Nonproprietary Names), USAN (United States Adopted
Names), IUPAC (International Union of Pure and Applied Chemistry). 
\item{\texttt{I\_UNII}, \texttt{I\_CAS}, \texttt{I\_CID}, \texttt{I\_CODE}.}
These stitch keys represent (i) unique identifiers assigned to the entity by
a well-known registrar (e.g., the U.S. Food and Drug Administration in
the case of UNII) or (ii) internal company code. \texttt{I\_UNII},
\texttt{I\_CAS}, and \texttt{I\_CID} are specific to drug (or
substance in general) entity type, whereas \texttt{I\_CODE} can be
used for any type of identifiers. The decision to use specific stitch
keys over generic ones ultimately rests on the strategies used for
entity resolution.
\item{\texttt{H\_LyChI\_L5}, \texttt{H\_LyChI\_L4}, \texttt{H\_LyChI\_L3}.}
For the small molecule class of drugs, perhaps more important than any
identifiers is the underlying chemical structure definition. These
stitch keys are hash values derived from the molecular structure at
different resolutions \citep{lychi}. Section~\ref{sec:methods-ingest}
discusses in detail how these derived stitch values are generated. 
\item{\texttt{R\_activeMoiety}.} Technically not a stitch key, the
active moiety relationship between two drugs provides a strong
evidence of equivalence. While this relationship can be infered
directly from the chemical structures (e.g., freebase and salt forms,
with and without esters), there is some level of curation needed to
handle structures with metal complex. 
\end{unlist}
Table~\ref{tab:imatinib} shows an example of stitch keys and stitch values for
the drug entity \emph{imatinib mesylate}. In this example,
the \texttt{R\_activeMoiety} relationship specifies the UNII of the
freebase form of imatinib mesylate. 

\begin{table}[thb]
\processtable{Stitch keys and stitch values for the
drug \emph{imatinib mesylate}\label{tab:imatinib}}
{\begin{tabular}{@{}ll@{}}\toprule Stitch key &
Stitch value\\\midrule
\texttt{N\_Name} & \texttt{IMATINIB MESYLATE}; \texttt{GLEEVEC}; \texttt{GLIVEC}\\
\texttt{I\_UNII} & \texttt{8A1O1M485B}\\
\texttt{I\_CAS} & \texttt{220127-57-1}\\
\texttt{I\_CID} & \texttt{5291}\\
\texttt{I\_CODE} & \texttt{STI-571}; \texttt{CHEMBL941}\\
\texttt{H\_LYCHI\_L5} & \texttt{7S4GKGNQ6N3X-N}\\
\texttt{H\_LYCHI\_L4} & \texttt{VLU17BQBSGWU-N}; \texttt{K83X3L3XSSHK-S}\\
\texttt{H\_LYCHI\_L3} & \texttt{VL3FPUQ59CU-N}; \texttt{K846NBMB7T3-S}\\
\texttt{R\_activeMoiety} & \texttt{BKJ8M8G5HI}\\\botrule
\end{tabular}}{}
\end{table}

\subsection{Data sources}
\st\ utilizes a number of diverse data sources for the \ix{} resource.
Among the data sources, of particular importance is the public G-SRS
data source from the FDA \citep{GSRSData}. This data source is
well-curated and contains over 100K substances across six different
classes: chemical, structurally diverse, protein, mixture, polymer,
and nucleic acid. As a data source derived from the FDA's internal
substance registry system, the G-SRS data source naturally forms the
basis of our data integration effort. The complete list of data
sources currently used by \st\ is shown in
Table~\ref{tab:data-sources}. 

\begin{table}[thb]
\processtable{\label{tab:data-sources}}
{\begin{tabular}{@{}ll@{}}\toprule
Data source & Size\\ \midrule
G-SRS, April 2019&	105,019\\
Withdrawn and Shortage Drugs List Feb 2018 &	674\\
Broad Institute Drug List 2018-09-07 &	6,125\\
NCATS Pharmaceutical Collection, April 2012 &	14,814\\
Rancho BioSciences, March 2019 &	51,591\\
Pharmaceutical Manufacturing Encyclopedia (Third Edition) &	2,268\\
DailyMed Rx, January 2019 &	74,850\\
DrugBank, December 2018&	11,922\\
DailyMed Other, January 2019&	13,393\\
DailyMed OTC, January 2019&	79,448\\
Drugs\@FDA \& Orange Book, July 2019&	28,256\\
ClinicalTrials, December 2017&	305,833\\
OTC Monographs, December 2018&	2,713\\
FDA NADA and ANADAs, December 2018&	554\\
FDA Excipients, December 2018&	10,212\\ \botrule
\end{tabular}}{}
\end{table}


\subsection{Overall strategy}
The basic premise behind \st{} is that data integration is often done
within the context of a specific data source. This is a reasonable
assumption given the data quality varies when integrating across
disparate sources. Futhermore, by establishing a reference data source
for data integration, we have finer control over the following: 
\begin{unlist}
\item{\emph{Data quality}.} A reference data source is typically
selected such that it is of high quality. Here, we can also impose
other data quality constraints (e.g., no synonyms can span multiple
entities) to guide entity resolution. 
\item{\emph{Data resolution}.} Entity resolution is particularly
challenging when data integration involves ontologies. A reference
data source can serve as the anchor ontology from which other
ontologies can be mapped. As with data quality, we can also impose any
additional semantic constraints; e.g., prostate cancer is not one of
the diagnoses for a female patient in an electronic health record. 
\item{\emph{Data curation}.} Generating ground-truth data is more
managable with a single data source than across multiple data sources.
This is particularly important due to the iterative feedback between
data curation and data integration. 
\end{unlist}
The G-SRS data source serves as an ideal reference data source. Its
rigorous substance models and well-structured data elements give us a
good starting point for drug data integration. In the next section, we
discuss our strategies in utilizing the G-SRS reference data source to
address entity resolution for drug data. 

\begin{methods}
\section{Methods}\label{sec:methods}
In general, data integration with \st{} consists of four basic
steps: \emph{ingestion}, \emph{stitching}, \emph{entity resolution},
and \emph{entity normalization}. With the exception of \emph{entity
resolution}, all other steps---as they are currently implemented
in \st---are generic and can be applied to a wide range of entity
types. 

\subsection{Data ingestion}\label{sec:methods-ingest}
\st{} is capable of ingesting data in a wide variety of sources and
formats. Semantic formats such as OWL, RDF, and Turtle are supported
as are JSON, delimiter separated text, and custom formats. For
non-semantic format, a separate configuration file is required to map
properties to stitch keys. 

An important step in data ingestion is the standaridzation and
validation of stitch values. For \texttt{N\_Name} stitch key, the
standardization procedure is simply to convert the input string to
uppercase; no validation is performed. For \texttt{I\_UNII}
and \texttt{I\_CAS} stitch keys, no standardization is required, and
validation is a simple checksum calculation to ensure the stitch value
is proper. Depending on the input format, \st\ also provides basic
utilities (e.g., regular expression) to help with data transformation
during ingestion. 

Perhaps the most unique feature of \st\ is its ability to incorporate
knowledge of chemical structures into entity resolution. Whereas
traditional approaches rely on names and identifiers to determine
equivalence relation among substances, \st\ goes a step further and
utilizes the underlying chemical structures to infer equivalence. This
is particularly relevant when the drug is a mixture, prodrug, or
active moiety with complex excipient (or derivative thereof). As an
example, consider the drug entity \emph{IMATINIB MESYLATE} and its
active ingredient \emph{IMATINIB}. Here, it is obvious that the two
entities cannot be matched by name alone. Instead, having structural
information by way of molecular hash keys for each molecular component
allows us to determine equivalence from the common active
moiety \emph{IMATINIB} between the two entities. This trivial example
might suggest that, instead of comparing names exactly, we find the
longest common substring of the names. The approach would certainly
work in this example, but to make it work in general would require
very specialized parsing rules and dictionaries. 

For data sources with chemical structures, the most computationally
demanding step in data ingestion is the generation of molecular hash
keys. Hash keys are generated for each component of a chemical
structure in three different structural
levels: \texttt{L5}, \texttt{L4}, and \texttt{L3}, which correspond to
stitch keys \texttt{H\_LYCHI\_L5}, \texttt{H\_LYCHI\_L4},
and \texttt{H\_LYCHI\_L3}, respectively. Level \texttt{L5} is the most
specific; it represents the chemical structure as-is, i.e., without
structure normalization and standardization. With the exception of the
relation \texttt{R\_activeMoiety}, a match at this level has higher
priority over other stitch keys. The next level \texttt{L4} represents
the structure after normalization and standardization per the LyChI
software package \citep{lychi}. A match at this level implies that two
structures are equivalent 
%insofar as \textcolor{red}{the valence bond theory is valid}
in terms of stereochemistry, resonance, and tautomerism. And the last
level \texttt{L3} is the same as \texttt{L4} but without
stereochemistry. A match at this level is considered weak and does not
constitute equivalence without other significant supporting evidence.
The purpose for \texttt{L3} is in anticipation of incorrect or missing
stereo information, which is one of the most common type of errors
associated with chemical structures. For each hash key, a
suffix \texttt{-M}, \texttt{-S}, or \texttt{-N} is also assigned to
designate the molecular component as either
a \emph{metal}, \emph{salt}, or \emph{neither}, respectively.
Table~\ref{tab:imatinib} illustrates all three representations for the
drug \emph{IMATINIB MESYLATE}. Note that the cardinality
for \texttt{L5} is always one, whereas for \texttt{L4} and \texttt{L3}
the cardinality is equal to the number of non-hydrate molecular
components. (Hydrate components are removed prior to processing.) 

\subsection{Data stitching}
\emph{Stitching} is the process by which the stitch multigraph is
incrementally constructed as data is ingested.
Algorithm~\ref{algo:stitching} describes the basic stitching algorithm
of \st. This algorithm is applied to each data source, and upon its
completion produced a stitch multigraph such that any stitch value
that spans $N$ stitch nodes is an induced clique, i.e., a complete
subgraph of $N$ nodes and $\frac{N(N-1)}{2}$ edges. Overlapping
induced cliques form the basis for the proposed entity resolution
approach discussed in the next section. As a side-effect, the
stitching algorithm also utilizes the union-find
algorithm \citep{Cormen2001} to efficiently track connected
components. 

\begin{algorithm}\label{algo:stitching}
\SetKwInOut{Input}{Input}
\SetKwInOut{Output}{Output}
\SetKwFunction{Union}{Union}
\SetKwFunction{Find}{Find}
\SetAlgoLined
\DontPrintSemicolon
Let $W$ denote the set of stitch nodes created in the data ingestion
step for a given data source $D$.\\
Let $\langle k, v\rangle$ be the tuple of stitch key and value,
respectively, defined for a stitch node $w$.\\
$G=(V,E)$ is current stitch multigraph.\\
\Find{$k, v$} is a function that returns all stitch nodes in $V$
containing stitch key $k$ and stitch value $v$.\\
\Union{$w,z$} is a union-find algorithm for tracking disjoint sets
(i.e., connected components).\\
 \For{$w \in W$}{
    \For{$\langle k_i, v_i\rangle \in w$}{
      \For{$z \in$ \Find{$k_i,v_i$}}{
         $E \leftarrow E \cup z\thicksim w$\;
         \Union{$w,z$}\;
      }
    }
    $V \leftarrow V \cup w$\;
 }
 \caption{Entity stitching algorithm}
\end{algorithm}

\subsection{Entity resolution}\label{sec:methods-er}
After all data sources have been stitched together, the next step is
to identify unique entities from the stitch multigraph. Formally, this
step is known as \emph{entity resolution} and is the only step
within \st\ that is specific to the drug entity type. This is to be
expected: Given that entity resolution is about adjudicating
the splitting and merging of entities, a reasonable amount of
knowledge of the entity type is required for the adjudication to be
effective. At the core of our proposed approach to entity resolution
for drug data is the reference G-SRS data source \citep{GSRSData}.
Using this data source as the ``seed'' from which other data sources
can map onto has several benefits:
\begin{itemize}
\item Since the G-SRS data source implements the ISO 11238
standard \citep{ISO11238} for defining medicinal substances, it serves
as an ideal starting point for defining what constitute a ``drug.''
\end{itemize}

The input to entity resolution is a connected component. Within \st\
the process of entity splitting and merging for a connected componoent
is known as \emph{untangling}. At a high level, this process works as
follows. 

\begin{algorithm}\label{algo:untangle}
\SetKwFunction{Union}{Union}
\SetKwFunction{Find}{Find}
\SetAlgoLined
\DontPrintSemicolon
$G=(V,E)$ is current stitch multigraph.\\
 \For{$w \in W$}{
    \For{$\langle k_i, v_i\rangle \in w$}{
      \For{$z \in$ \Find{$k_i,v_i$}}{
         $E \leftarrow E \cup z\thicksim w$\;
         \Union{$w,z$}\;
      }
    }
    $V \leftarrow V \cup w$\;
 }
 \caption{An algorithm to untangle a connected component}
\end{algorithm}


\subsection{Entity normalization}
Text describing entity normalization

%\enlargethispage{6pt}
\end{methods}

\section{Results}
Discussion of results and highlight interesting examples here.

\section{Discussion}

\section*{Acknowledgements}
We thank our colleagues, Mark Williams and Tyler Beck, for their
valuable proof-reading of early drafts of the manuscript. We also
thank our colleague, Tongan Zhao, for his help in developing a
prototype curation user interface for \st.

\bibliographystyle{natbib}
%\bibliographystyle{achemnat}
%\bibliographystyle{plainnat}
%\bibliographystyle{abbrv}
%\bibliographystyle{bioinformatics}
%
%\bibliographystyle{plain}
%
\bibliography{stitcher}

\end{document}
